\documentclass[../main.tex]{subfiles}
\begin{document}

\chapter{Boundary-Value Problems}

\begin{center}
    \Large{\textbf{CHAPTER OBJECTIVES}}
\end{center}
The primary objective of this chapter is to introduce you to solving boundary-value
problems for ODEs. Specific objectives and topics covered are
\begin{itemize}
    \item Understanding the difference between initial-value and boundary-value problems
    \item Knowing how to express an nth-order ODE as a system of n first-order ODEs.
    \item Knowing how to implement the shooting method for linear ODEs by using linear
    interpolation to generate accurate ``shots.''
    \item Understanding how derivative boundary conditions are incorporated into the
    shooting method.
    \item Knowing how to solve nonlinear ODEs with the shooting method by using root
    location to generate accurate ``shots.''
    \item Knowing how to implement the finite-difference method.
    \item Understanding how derivative boundary conditions are incorporated into the
    finite-difference method.
    \item Knowing how to solve nonlinear ODEs with the finite-difference method by using
    root-location methods for systems of nonlinear algebraic equations.
\end{itemize}

\newpage

\large{\textbf{You've got a problem.}}

\noindent To this point, we have been computing the velocity of a free-falling bungee jumper by
integrating a single ODE:

\begin{equation}
    \tag{24.1}
    \frac{d v}{d t}=g-\frac{c_{d}}{m} v^{2}
\end{equation}

Suppose that rather than velocity, you are asked to determine the position of the jumper as
a function of time. One way to do this is to recognize that velocity is the first derivative of distance:

\begin{equation}
    \tag{24.2}
    \frac{dx}{dt} = v
\end{equation}

\noindent Thus, by solving the system of two ODEs represented by Eqs. (24.1) and (24.2), we can
simultaneously determine both the velocity and the position.

However, because we are now integrating two ODEs, we require two conditions to
obtain the solution. We are already familiar with one way to do this for the case where we
have values for both position and velocity at the initial time:
\end{document}